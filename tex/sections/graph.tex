\section{PageRank Algorithm}
We are interested in calculating PageRank before calculating it, we need to know about the topology of the web.\\
\\
In order to do that, we need to view the Web as a graph. The Web's hyperlink structure forms a massive directed graph. The nodes in the graph represent webpages and the directed arcs or links represent the hyperlinks. Thus, hyperlinks into a page, which are called inlinks, point into nodes, while outlinks point out from nodes. Nodes with no outlinks are known as dangling node. We will see how it causes the PageRank.\\
	
\begin{center}
\begin{tikzpicture}
\begin{scope}[every node/.style={circle,thick,draw}]
	\node (P1) at (0,3) {P1};
	\node (P3) at (2.5,1) {P3};
	\node (P2) at (5,3) {P2} ;
	\node (P4) at (0,6) {P4};
	\node (P5) at (2.5,8) {P5};
	\node (P6) at (5,6) {P6} ;
\end{scope}			
\begin{scope}[>={Stealth[black]},
every node/.style={fill=white,circle},
every edge/.style={draw=cyan,very thick}]
\path [->] (P1) edge[bend left=20] (P2);
\path [->] (P1) edge[bend left=-20] (P3);
\path [->] (P1) edge[bend left=0] (P4);
\path [->] (P2) edge[bend left=20] (P1);
\path [->] (P2) edge[bend left=20] (P3);
\path [->] (P2) edge[bend left=0] (P6);
\path [->] (P4) edge[bend left=0] (P6);
\path [->] (P4) edge[bend left=0] (P5);
\path [->] (P5) edge[bend left=20] (P6);
\path [->] (P6) edge[bend left=0] (P1);
\path [->] (P6) edge[bend left=20] (P5);
\end{scope}
\end{tikzpicture}
\end{center}

A page is important if it is pointed to by other important pages.

\noindent Let's define PageRank of a page $P_{i}$, denoted $r\left(P_{i}\right)$ and it is the sum of the PageRanks of all pages pointing into $P_{i}$.
	
$$
r\left(P_{i}\right)=\sum_{P_{j} \in B_{P_{i}}} r\left(P_{j}\right)
$$
$B_{P_{i}}:$ is the set of pages pointing into $P_{i}$ \\
\\
The problem with formula is that the $r(P_{j})$ values, the PageRanks of pages inlinking to page $P_{i}$, are unknown. To sidestep this problem, we are going to use an iterative approach(process). That is, we are assuming that, in the beginning, all pages have equal PageRank (say, $1/n$, here $n$ is the number of web pages). As we are applying the rule in formula successively, we need to introduce some more notation in order to distinguish steps. Let $r_{k}\left(P_{i}\right)$ be the PageRank of page $P_{i}$ at $k^{th}$iteration. Then, PageRank of $P_i$ at $(k+1)^th$ iteration is given by
	
$$
r_{k+1}\left(P_{i}\right)=\sum_{P_{j} \in B_{P_{i}}} r_{k}\left(P_{j}\right)
$$
\begin{center}
\begin{tikzpicture}
\begin{scope}[every node/.style={circle,thick,draw}]
\node (P4) at (2.5,3) {P4};
\node (P2) at (0,5.5) {P2};
\node (P1) at (2.5,8) {P1};
\node (P3) at (5,5.5) {P3} ;
\end{scope}
			
\begin{scope}[>={Stealth[black]},
every node/.style={fill=white,circle},
every edge/.style={draw=cyan,very thick}]
\path [->] (P1) edge[bend left=0] (P2);
				\path [->] (P2) edge[bend left=0] (P4);
				\path [->] (P2) edge[bend left=0] (P3);
				\path [->] (P4) edge[bend left=0] (P3);
				\path [->] (P3) edge[bend left=0] (P1);
			\end{scope}
		\end{tikzpicture}
	\end{center}
	
\noindent This process is initiated with $r_{0}\left(P_{i}\right)=1 / n$ for all pages $P_{i}$ and repeated with the hope that the PageRank scores will eventually converge to some final stable values. Applying equation to the above tiny web of 4-nodes gives the following values for the PageRanks after a few iterations.
\\

\begin{table}[h]
	\centering
	\begin{tabular}{l l l c}
		\toprule
		\textbf{Iteration 0} & \textbf{Iteration 1} & \textbf{Iteration 2} & \textbf{Iteration 3}\\
		\midrule
		$r_{0}\left(P_{1}\right)=1 / 4$ & $r_{1}\left(P_{1}\right)=1 / 4$ & $r_{2}\left(P_{1}\right)=2 / 4$ & 
		$r_{3}\left(P_{1}\right)=2 / 4$\\
		$r_{0}\left(P_{2}\right)=1 / 4$ & $r_{1}\left(P_{2}\right)=1 / 4$ & $r_{2}\left(P_{2}\right)=1 / 4$ & 
		$r_{3}\left(P_{2}\right)=2 / 4$\\
		$r_{0}\left(P_{3}\right)=1 / 4$ & $r_{1}\left(P_{3}\right)=2 / 4$ & $r_{2}\left(P_{3}\right)=2 / 4$ & 
		$r_{3}\left(P_{3}\right)=3 / 4$\\
		$r_{0}\left(P_{4}\right)=1 / 4$ & $r_{1}\left(P_{4}\right)=1 / 4$ & $r_{2}\left(P_{4}\right)=1 / 4$& 
		$r_{3}\left(P_{4}\right)=1 / 4$\\
		\bottomrule
	\end{tabular}
\end{table}

\noindent From the above table we can say that Page $P_3$ has highest rank and Page $P_4$ has lowest rank.\\
\\
\noindent This algorithm can be exploited by SEO's (Search Engine Optimization) by just creating dummy websites and pointing(more formally by giving backlinks) to their clients.\\
\\
\noindent This problem can be overcome by little change in previous algorithm.

$$
r\left(P_{i}\right)=\sum_{P_{j} \in B_{P_{i}}} \frac{r\left(P_{j}\right)}{\left|P_{j}\right|}
$$
$B_{P_{i}}:$ is the set of pages pointing into $P_{i}$ \\
$|P_{j}|:$ is the number of outlinks from page $P_{j}$.\\
\\
Recursive formula:
$$
r_{k+1}\left(P_{i}\right)=\sum_{P_{j} \in B_{P_{i}}} \frac{r_{k}\left(P_{j}\right)}{\left|P_{j}\right|} .
$$

\noindent Now, unlike previous algorithm dummy websites have no use as their value of backlinks decreases as the number of backlinks increase.

\begin{center}
	\begin{tikzpicture}
		\begin{scope}[every node/.style={circle,thick,draw}]
			\node (P1) at (0,3) {P1};
			\node (P3) at (2.5,1) {P3};
			\node (P2) at (5,3) {P2} ;
			\node (P4) at (0,6) {P4};
			\node (P5) at (2.5,8) {P5};
			\node (P6) at (5,6) {P6} ;
		\end{scope}
		\begin{scope}[>={Stealth[black]},
			every node/.style={fill=white,circle},
			every edge/.style={draw=cyan,very thick}]
			\path [->] (P1) edge[bend left=20] (P2);
			\path [->] (P1) edge[bend left=-20] (P3);
			\path [->] (P1) edge[bend left=0] (P4);
			\path [->] (P2) edge[bend left=20] (P1);
			\path [->] (P2) edge[bend left=20] (P3);
			\path [->] (P2) edge[bend left=0] (P6);
			\path [->] (P4) edge[bend left=0] (P6);
			\path [->] (P4) edge[bend left=0] (P5);
			\path [->] (P5) edge[bend left=20] (P6);
			\path [->] (P6) edge[bend left=0] (P1);
			\path [->] (P6) edge[bend left=20] (P5);
		\end{scope}
	\end{tikzpicture}
\end{center}
	
Let's calculate PageRank using updated algorithm,
	
\begin{table}[h]
	\centering
	\begin{tabular}{l l l c}
		\toprule
	\textbf{Iteration 0} & \textbf{Iteration 1} & \textbf{Iteration 2} & \textbf{PageRank}\\
		\midrule
		$r_{0}\left(P_{1}\right)=1 / 6$ & $r_{1}\left(P_{1}\right)=5 / 36$ & $r_{2}\left(P_{1}\right)=37 / 216$ & 3 \\
		$r_{0}\left(P_{2}\right)=1 / 6$ & $r_{1}\left(P_{2}\right)=2 / 36$ & $r_{2}\left(P_{2}\right)=10 / 216$ & 5 \\
		$r_{0}\left(P_{3}\right)=1 / 6$ & $r_{1}\left(P_{3}\right)=4 / 36$ & $r_{2}\left(P_{3}\right)=14 / 216$ & 4 \\
		$r_{0}\left(P_{4}\right)=1 / 6$ & $r_{1}\left(P_{4}\right)=2 / 36$ & $r_{2}\left(P_{4}\right)=10 / 216$ & 5 \\
		$r_{0}\left(P_{5}\right)=1 / 6$ & $r_{1}\left(P_{5}\right)=6 / 36$ & $r_{2}\left(P_{5}\right)=39 / 216$ & 2 \\
		$r_{0}\left(P_{6}\right)=1 / 6$ & $r_{1}\left(P_{6}\right)=11 / 36$ & $r_{2}\left(P_{6}\right)=46 / 216$ & 1\\			\bottomrule
		\end{tabular}
\end{table}